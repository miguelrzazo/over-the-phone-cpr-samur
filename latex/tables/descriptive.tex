\begin{table}[ht]
\centering
\small
\caption{Características de las paradas cardiorrespiratorias estudiadas en función del tipo de asistencia recibida antes de la llegada del primer recurso asistencial medicalizado.}
\resizebox{\textwidth}{!}{ % Ajusta la tabla al ancho de la página
\begin{tabular}{l c c c c c}
\arrayrulecolor{black}\hline
& \textbf{Total} & \textbf{Sin RCP previa} & \multicolumn{3}{c}{\textbf{RCP previa iniciada}} \\ 
& \textbf{(n=596)} & \textbf{(n=530)} & \textbf{No guiada por tlf} & \textbf{Guiada por tlf} & \textbf{Primer respondiente} \\ 
& & & \textbf{(n=52)} & \textbf{(n=14)} & \textbf{(n=0)} \\ 
\arrayrulecolor{gray}\cline{4-6}
\textit{Mujeres, n (\%)} & 134 (22.5) & 119 (22.5) & 11 (21.2) & 4 (28.6) & -- (--) \\ \arrayrulecolor{black}\hline
\textit{Edad (años), med (RIC)} & 64 [54-77] & 64 [54-77] & 64 [54-77] & 74 [58-84] & -- [--,--] \\ \hline
\textit{Edad $\geq$ 65 años, n (\%)} & 271 (45.5) & 238 (44.9) & 25 (48.1) & 8 (57.1) & -- (--) \\ \hline
\textit{PCR presenciada, n (\%)} & 453 (76.0) & 410 (77.4) & 31 (59.6) & 12 (85.7) & -- (--) \\ \hline
\textit{Ritmo inicial no desfibrilable, n (\%)} & 266 (44.6) & 223 (42.1) & 34 (65.4) & 9 (64.3) & -- (--) \\ \hline
\textit{Uso de DESA, n (\%)} & 70 (11.7) & -- & 9 (17.3) & 1 (7.1) & -- (--) \\ \hline
\textit{\begin{tabular}[c]{@{}l@{}}Tiempo 1a USVA (min)\end{tabular}} & 7.3 [5.3-9.9] & 7.2 [5.2-9.9] & 8.0 [5.9-10.2] & 6.9 [5.7-9.7] & -- [--,--] \\ \hline
\multicolumn{6}{l}{\textbf{Evolución clínica, n (\%)}} \\ \hline
\textit{ROSC} & 297 (49.8) & 261 (49.2) & 28 (53.8) & 8 (57.1) & -- (--) \\ \hline
\textit{Supervivencia a los 7 días} & 92 (15.4) & 78 (14.7) & 11 (21.2) & 3 (21.4) & -- (--) \\ \hline
\textit{Alta con CPC 1-2} & 60 (10.1) & 53 (10.0) & 5 (9.6) & 2 (14.3) & -- (--) \\ \hline
\textit{Alta con CPC 3-4} & 2 (0.3) & 2 (0.4) & -- (--) & -- (--) & -- (--) \\ \hline
\end{tabular}
}
\begin{flushleft}
    
\scriptsize
DESA: desfibrilador externo semiautomático;
CPC: condición clínica del paciente evaluada según la Escala Glasgow–Pittsburgh Cerebral Performance Categories;
PCR: parada cardiorrespiratoria; RCP: reanimación cardiopulmonar;
SEM: Servicio de Emergencias Médicas. Primer respondiente incluye a SVB, personal hospital, socorristas, bomberos y policía.
ROSC: retorno de circulación espontánea. Los valores expresan $n$ (\%) o mediana [intervalo intercuartílico].

\end{flushleft}
\label{tab:descriptive_statistics}
\end{table}
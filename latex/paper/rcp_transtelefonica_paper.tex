\documentclass[10pt,a4paper]{article}
\usepackage[utf8]{inputenc}
\usepackage[spanish]{babel}
\usepackage{amsmath}
\usepackage{amsfonts}
\usepackage{amssymb}
\usepackage{graphicx}
\usepackage{float}
\usepackage{geometry}
\usepackage{booktabs}
\usepackage{longtable}
\usepackage{array}
\usepackage{multirow}
\usepackage{xcolor}
\usepackage{hyperref}
\usepackage{authblk}
\usepackage{siunitx}
\usepackage{caption}
\usepackage{multicol}
\usepackage{xcolor}
\usepackage{colortbl}
% Set sans-serif as the main font
\renewcommand{\familydefault}{\sfdefault}
\renewcommand{\tablename}{Tabla}

% Configuración de página
\geometry{margin=2.5cm}
% Set interlineado a 1
\linespread{1}

% Configuración de hyperref
\hypersetup{
    colorlinks=true,
    linkcolor=blue,
    filecolor=magenta,      
    urlcolor=cyan,
    citecolor=red
}

% Título y autores
\title{\textbf{Efectividad de la Reanimación Cardiopulmonar Transtelefónica en Paradas Cardiacas Extrahospitalarias: Análisis Retrospectivo de SAMUR-PC 2023-2025}}

\author[1]{Maria del Rosario Muñoz Condés\thanks{email@email.com}}
\author[1]{Miguel Rosa Zazo}
\author[1]{Óscar Córcoba Fernández}
\author[1]{Autor 4}
\author[1]{Autor 5}
\author[1]{Autor 6}
\affil[1]{SAMUR-PC, Madrid, España}

\date{\today}

\begin{document}

\maketitle

\noindent\textbf{Palabras clave:} Parada cardiaca extrahospitalaria, RCP transtelefónica, ROSC, supervivencia, servicios de emergencia médica

\begin{abstract}
\textbf{Introducción:} La parada cardiaca extrahospitalaria (PCEH) es una de las principales causas de mortalidad a nivel mundial. La RCP transtelefónica, en la que los operadores de emergencias guían a los testigos a través de instrucciones telefónicas, ha surgido como una estrategia clave para mejorar los resultados en pacientes con PCEH. Este estudio analiza su efectividad en comparación con otros tipos de RCP.

\textbf{Objetivos:} Evaluar el impacto de la RCP transtelefónica en la supervivencia, el retorno de circulación espontánea (ROSC) y el estado neurológico favorable de pacientes en PCR en la ciudad de Madrid.

\textbf{Metodología:} Análisis observacional retrospectivo de casos de PCEH atendidos por SAMUR-PC entre 2023 y 2025. Se compararon grupos según el tipo de RCP recibida: sin RCP previa, RCP no guiada por teléfono, RCP guiada por teléfono y RCP por primer respondiente. Se utilizaron análisis estadísticos como regresión logística, curvas de Kaplan-Meier y modelos de Cox, ajustando por factores de confusión.


\textbf{Resultados:} 


\textbf{Conclusiones:} 

\end{abstract}


\newpage

\begin{multicols}{2}

\subsection*{Introducción}

La parada cardiaca extrahospitalaria (PCEH) representa una de las principales causas de mortalidad a nivel mundial, con tasas de supervivencia que varían ampliamente dependiendo de factores como la rapidez en la intervención y la calidad de la reanimación cardiopulmonar (RCP). En este contexto, la RCP transtelefónica, en la que los operadores de emergencias guían a los testigos a través de instrucciones telefónicas, ha emergido como una estrategia clave para mejorar los resultados en pacientes con PCEH. Sin embargo, persisten interrogantes sobre su efectividad en comparación con otros tipos de RCP, especialmente en diferentes grupos de edad y tiempos de respuesta. Este estudio tiene como objetivo evaluar el impacto de la RCP transtelefónica en la supervivencia y el estado neurológico de los pacientes atendidos por el Servicio de Asistencia Municipal de Urgencias y Rescate de Madrid (SAMUR-PC).


\subsection*{Metodología}

\subsubsection*{Diseño del estudio}
El estudio se basa en un análisis observacional retrospectivo de casos de parada cardiaca extrahospitalaria (PCR) atendidos por el Servicio de Asistencia Municipal de Urgencias y Rescate de Madrid (SAMUR-PC) entre el 1 de julio de 2023 y el 30 de junio de 2025. Estos datos son obtenidos mediante el informe electrónico cumplimentado por las unidades de SAMUR-PC, asi como los formularion rellenados por los teleoperadores de la central de comunicaciones del mismo servicio.
De la muestra original se excluyen los casos de paradas traumáticas. También se excluyen las filas que no tienen información sobre RCP transtelefónica. En los casos en que los datos de soporte vital avanzado (SVA) y soporte vital básico (SVB) coinciden en fecha y hora, se fusionan priorizando los datos de SVA. Si no es posible emparejar los datos de SVB, estos se eliminan. 

\subsubsection*{Descripcion de la muestra}


La muestra se divide en los siguientes grupos a valorar:

(1) \textbf{Sin RCP previa:} Pacientes que no recibieron RCP antes de la llegada del primer recurso asistencial medicalizado.
\textbf{RCP previa iniciada:} Pacientes que recibieron RCP antes de la llegada del primer recurso asistencial medicalizado, subdivididos en:
(2) \textbf{No guiada por teléfono:} RCP iniciada sin asistencia telefónica.
(3) \textbf{Guiada por teléfono:} RCP iniciada con asistencia telefónica. 
(4) \textbf{Primer respondiente:} RCP iniciada por el primer respondiente, que puede incluir personal de SVB, personal hospitalario, socorristas, bomberos o policía.

Todos ellos son comparados al grupo de control sin RCP previa, para valorar su efectividad. La \hyperref[tab:descriptive_statistics]{Tabla \ref*{tab:descriptive_statistics}} describe la muestra a evaluar.


\subsubsection*{Criterios de valoración}

Los grupos listados se comparan y valoran según los siguientes criterios:
(1) \textbf{ROSC:} Retorno de circulación espontánea.
(2) \textbf{Supervivencia a los 7 días:} Pacientes que sobrevivieron al menos 7 días tras la PCR.
(3) \textbf{Alta con CPC 1-2:} Pacientes dados de alta con una condición clínica favorable según la Escala Glasgow–Pittsburgh Cerebral Performance Categories.

\subsubsection*{Análisis estadístico}
Para las variables categóricas, se realizaron análisis de tablas de contingencia utilizando la prueba de Chi-cuadrado para evaluar asociaciones entre los grupos de estudio y las variables de resultado. En los casos en que las frecuencias esperadas en alguna celda fueran inferiores a 5, se utilizó la prueba exacta de Fisher para garantizar la validez estadística. 

Adicionalmente, se emplearon modelos de regresión logística para estimar odds ratios (OR) ajustados, junto con sus respectivos intervalos de confianza al 95\%, con el objetivo de cuantificar la asociación entre el tipo de RCP y los resultados principales, ajustando por posibles factores de confusión como edad, sexo, tiempo de llegada de los servicios de emergencia y comorbilidades.

Para las variables continuas, se aplicaron modelos de regresión lineal con el fin de evaluar asociaciones y tendencias entre los grupos de estudio y las variables de resultado. En caso de que las distribuciones no cumplieran con los supuestos de normalidad, se utilizaron pruebas no paramétricas como la prueba de Mann-Whitney U para comparar medianas entre grupos.

En cuanto al análisis de supervivencia, se generaron curvas de Kaplan-Meier para comparar las tasas de supervivencia entre los grupos de estudio, utilizando la prueba de log-rank para determinar diferencias estadísticamente significativas. Además, se implementaron modelos de riesgos proporcionales de Cox para estimar hazard ratios (HR) ajustados, considerando factores de confusión como edad, sexo, tiempo de llegada y tipo de RCP.

Todos los análisis estadísticos se realizaron utilizando el entorno de programación Python, empleando las librerías pandas, numpy, scipy y statsmodels para la manipulación de datos y pruebas estadísticas. Los gráficos se generaron con matplotlib y seaborn, siguiendo un diseño visual uniforme y científico. El código fuente completo, junto con los scripts de análisis, está disponible en el repositorio público de GitHub\footnote{\url{https://github.com/miguelrzazo/over-the-phone-cpr-samur}} para garantizar la reproducibilidad del estudio.


\end{multicols}
\onecolumn

\begin{table}[ht]
\centering
\small
\caption{Características de las paradas cardiorrespiratorias estudiadas en función del tipo de asistencia recibida antes de la llegada del primer recurso asistencial medicalizado. DATOS PLACEHOLDER}
\resizebox{\textwidth}{!}{ % Ajusta la tabla al ancho de la página
\begin{tabular}{l c c c c c}
\arrayrulecolor{black}\hline
& \textbf{Total} & \textbf{Sin RCP previa} & \multicolumn{3}{c}{\textbf{RCP previa iniciada}} \\ 
& \textbf{(n=1.603)} & \textbf{(n=923)} & \textbf{No guiada por tlf} & \textbf{Guiada por tlf} & \textbf{Primer respondiente} \\ 
& & & \textbf{(n=407)} & \textbf{(n=273)} & \textbf{(n=XX)} \\ 
\arrayrulecolor{gray}\cline{4-6}
\textit{Mujeres, n (\%)}                        & 439 (27,4)   & 274 (29,7) & 96 (23,6)  & 69 (25,3)  & XX (XX) \\ \arrayrulecolor{black}\hline
\textit{Edad (años), med (RIC)}                 & 68 [56-79]   & 71 [59-81] & 65 [53-77] & 62 [52-73] & XX [XX-XX] \\ \hline
\textit{Edad $\geq$ 65 años, n (\%)}            & 917 (57,2)   & 595 (64,5) & 207 (50,9) & 115 (42,1) & XX (XX) \\ \hline
\textit{PCR presenciada, n (\%)}                & 279 (17,5)   & 147 (16,1) & 80 (19,7)  & 70 (25,6)  & XX (XX) \\ \hline
\textit{Ritmo inicial no desfibrilable, n (\%)}  & 1198 (74,7)  & 747 (80,9) & 255 (71)   & 196 (71,8) & XX (XX) \\ \hline
\textit{Uso de DESA, n (\%)}  & 52 (4,7)     & --         & 46 (6,8)   & 6 (2,4)    & XX (XX) \\ \hline
\textit{\begin{tabular}[c]{@{}l@{}}Tiempo 1a USVA (min)\end{tabular}} 
                                      & 9,0 [7,0-13,0] & 9,0 [7,0-13,0] & 9,0 [5,0-13,0] & 10,0 [7,0-14,0] & XX [XX-XX] \\ \hline
\multicolumn{6}{l}{\textbf{Evolución clínica, n (\%)}} \\ \hline
\textit{ROSC}                 & 1.150 (71,7)  & 679 (73,6) & 264 (64,9) & 207 (75,8) & XX (XX) \\ \hline
\textit{Supervivencia a los 7 días}         & 308 (19,2)    & 172 (18,6) & 87 (21,4)  & 49 (17,9)  & XX (XX) \\ \hline
\textit{Alta con CPC 1-2}                      & 134 (8,4)     & 69 (7,5)   & 53 (13)    & 12 (4,4)   & XX (XX) \\ \hline
\textit{Alta con CPC 3-4}                      & 11 (0,7)      & 3 (0,3)    & 3 (0,7)    & 5 (1,8)    & XX (XX) \\ \hline
\end{tabular}
}
\begin{flushleft}
    
\scriptsize
DESA: desfibrilador externo semiautomático;
CPC: condición clínica del paciente evaluada según la Escala Glasgow–Pittsburgh Cerebral Performance Categories;
PCR: parada cardiorrespiratoria; RCP: reanimación cardiopulmonar;
SEM: Servicio de Emergencias Médicas. Primer respondiente incluye a SVB, personal hospital, socorristas, bomberos y policía.
ROSC: retorno de circulación espontánea. Los valores expresan $n$ (\%) o mediana [intervalo intercuartílico].

\end{flushleft}
\label{tab:descriptive_statistics}
\end{table}
\begin{multicols}{2}


\subsection*{Resultados}



Limitaciones del estudio:


Mejorar el modelo de recolección de datos...
Hay que mejorar la formación a la ciudadanía, a los operadores...

\subsection*{Conclusión}



\subsection*{Conflicto de intereses}
Los autores declaran no tener conflictos de intereses en relación con este estudio.

\newpage

\begin{thebibliography}{99}


\end{thebibliography}

\end{multicols}

\end{document}
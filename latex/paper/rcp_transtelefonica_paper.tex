\documentclass[10pt,a4paper]{article}
\usepackage[utf8]{inputenc}
\usepackage[spanish]{babel}
\usepackage{amsmath}
\usepackage{amsfonts}
\usepackage{amssymb}
\usepackage{graphicx}
\usepackage{float}
\usepackage{geometry}
\usepackage{booktabs}
\usepackage{longtable}
\usepackage{array}
\usepackage{multirow}
\usepackage{xcolor}
\usepackage{hyperref}
\usepackage{authblk}
\usepackage{siunitx}
\usepackage{caption}
\usepackage{multicol}

% Set sans-serif as the main font
\renewcommand{\familydefault}{\sfdefault}

% Configuración de página
\geometry{margin=2.5cm}
% Set interlineado a 1
\linespread{1}

% Configuración de hyperref
\hypersetup{
    colorlinks=true,
    linkcolor=blue,
    filecolor=magenta,      
    urlcolor=cyan,
    citecolor=red
}

% Título y autores
\title{\textbf{Efectividad de la Reanimación Cardiopulmonar Transtelefónica en Paradas Cardiacas Extrahospitalarias: Análisis Retrospectivo de SAMUR-PC 2023-2025}}

\author[1]{Muñoz Condes, M.}
\author[1]{Rosa Zazo, M.}
\author[1]{Córcoba Fernández, O.}
\author[1]{Autor 4}
\author[1]{Autor 5}
\author[1]{Autor 6}
\affil[1]{SAMUR-PC, Madrid, España}

\date{\today}

\begin{document}

\maketitle

% Abstract with sections in Spanish
\begin{abstract}
\textbf{Introducción:} La parada cardiaca extrahospitalaria representa una de las principales causas de mortalidad, con tasas de supervivencia que varían significativamente según la rapidez y calidad de la reanimación cardiopulmonar (RCP) inicial. La RCP guiada telefónicamente por servicios de emergencias médicas se ha propuesto como una estrategia para mejorar los resultados clínicos.

\textbf{Objetivos:} Evaluar la efectividad de la RCP transtelefónica en la recuperación de circulación espontánea (ROSC) y supervivencia a 7 días en paradas cardiacas extrahospitalarias atendidas por SAMUR-PC.

\textbf{Metodología:} Estudio observacional retrospectivo de casos de parada cardiaca extrahospitalaria atendidos por SAMUR-PC entre 2023 y 2025. Se analizaron variables demográficas, tiempo de respuesta, tipo de RCP realizada y resultados clínicos. Se emplearon análisis de regresión logística multivariante, pruebas de chi-cuadrado y t de Student para evaluar la asociación entre RCP transtelefónica y los resultados primarios.

\textbf{Resultados:} [Los resultados específicos se completarán tras el análisis de datos.]

\textbf{Conclusiones:} [Las conclusiones se desarrollarán basándose en los resultados del análisis.]

\textbf{Palabras clave:} Parada cardiaca extrahospitalaria, RCP transtelefónica, ROSC, supervivencia, servicios de emergencia médica
\end{abstract}

\newpage

\begin{multicols}{2}

\section{Introducción}

La parada cardiaca extrahospitalaria (PCEH) constituye una emergencia médica de primer orden que afecta aproximadamente a 350,000-700,000 personas anualmente en Europa \cite{ref1}. A pesar de los avances en los cuidados intensivos y las técnicas de reanimación, la supervivencia al alta hospitalaria permanece baja, oscilando entre el 8-12\% en la mayoría de los sistemas de emergencias médicas \cite{ref2}.

La cadena de supervivencia en la PCEH identifica cuatro eslabones críticos: reconocimiento temprano y activación del sistema de emergencias, RCP temprana de alta calidad, desfibrilación precoz y cuidados post-reanimación avanzados \cite{ref3}. Entre estos, la RCP temprana realizada por testigos constituye uno de los factores pronósticos más importantes, pudiendo duplicar o triplicar las tasas de supervivencia cuando se inicia en los primeros minutos tras la parada \cite{ref4}.

En este contexto, la RCP guiada telefónicamente (T-CPR) por los servicios de emergencias médicas representa una intervención crucial para facilitar la RCP por testigos. Los operadores de emergencias, entrenados específicamente para ello, pueden proporcionar instrucciones paso a paso a los testigos presenciales, mejorando tanto la calidad como la rapidez del inicio de las maniobras de reanimación \cite{ref5}.

SAMUR-PC (Servicio de Asistencia Municipal de Urgencia y Rescate - Protección Civil) de Madrid ha implementado protocolos específicos para la guía telefónica de RCP desde hace varios años. Sin embargo, la efectividad real de esta intervención en el contexto específico de nuestro sistema requiere evaluación sistemática.

\section{Objetivos}

\subsection{Objetivo Principal}
Evaluar la efectividad de la RCP transtelefónica en la recuperación de circulación espontánea (ROSC) en paradas cardiacas extrahospitalarias atendidas por SAMUR-PC.

\subsection{Objetivos Secundarios}
\begin{itemize}
\item Analizar la asociación entre RCP transtelefónica y supervivencia a 7 días
\item Identificar factores demográficos y clínicos asociados con mejores resultados en casos con RCP transtelefónica
\item Evaluar el impacto del tiempo de respuesta en la efectividad de la RCP transtelefónica
\item Caracterizar el perfil epidemiológico de las paradas cardiacas extrahospitalarias en el área de cobertura de SAMUR-PC
\end{itemize}

\section{Metodología}

\subsection{Diseño del Estudio}
Estudio observacional retrospectivo de casos de parada cardiaca extrahospitalaria atendidos por SAMUR-PC durante el período comprendido entre enero de 2023 y diciembre de 2025.

\subsection{Población y Muestra}
Se incluyeron todos los casos de parada cardiaca extrahospitalaria confirmada atendidos por SAMUR-PC durante el período de estudio. Los criterios de inclusión fueron:
\begin{itemize}
\item Parada cardiaca confirmada de origen extrahospitalario
\item Edad mayor de 18 años
\item Disponibilidad de datos completos sobre el tipo de RCP realizada
\item Registro completo de variables de resultado principales
\end{itemize}

Se excluyeron casos con parada cardiaca de origen traumático, ahogamiento o intoxicación, así como aquellos con datos incompletos en las variables principales del estudio.

\subsection{Variables de Estudio}

\textbf{Variable de exposición principal:}
\begin{itemize}
\item RCP transtelefónica: Variable dicotómica que indica si se proporcionó guía telefónica para la realización de RCP por testigos
\end{itemize}

\textbf{Variables de resultado:}
\begin{itemize}
\item ROSC (Return of Spontaneous Circulation): Recuperación de circulación espontánea sostenida
\item Supervivencia a 7 días: Estado vital del paciente a los 7 días del evento
\end{itemize}

\textbf{Variables de confusión y estratificación:}
\begin{itemize}
\item Edad (años)
\item Sexo (masculino/femenino)
\item Tiempo de llegada de la unidad de emergencia (segundos)
\item Presencia de testigos
\item Ritmo cardiaco inicial
\end{itemize}

\subsection{Análisis Estadístico}

Se realizó un análisis descriptivo de todas las variables, expresando las variables categóricas como frecuencias y porcentajes, y las variables continuas como media y desviación estándar o mediana y rango intercuartílico según su distribución.

Para el análisis bivariado se emplearon:
\begin{itemize}
\item Prueba de chi-cuadrado para comparar proporciones entre grupos
\item Prueba t de Student para comparar medias entre grupos con distribución normal
\item Prueba U de Mann-Whitney para variables con distribución no normal
\end{itemize}

Se construyeron modelos de regresión logística multivariante para evaluar la asociación independiente entre RCP transtelefónica y los resultados principales, ajustando por variables de confusión. Los resultados se expresaron como odds ratios (OR) con intervalos de confianza del 95\%.

Se consideró estadísticamente significativo un valor p < 0.05. Todos los análisis se realizaron con Python 3.15 utilizando las librerías pandas, numpy, scipy y matplotlib.

\section{Resultados}

\subsection{Características Basales de la Población}

[Esta sección se completará con los resultados del análisis del notebook 08]

Durante el período de estudio se registraron un total de [N] casos de parada cardiaca extrahospitalaria que cumplieron los criterios de inclusión.

\textbf{Tabla 1. Características basales de la población de estudio}

[Aquí se insertará la tabla de características basales generada en el notebook]

\subsection{Análisis de Efectividad de la RCP Transtelefónica}

\subsubsection{Recuperación de Circulación Espontánea (ROSC)}

[Los resultados específicos del análisis de ROSC se insertarán aquí]

\textbf{Figura 1. Tasas de ROSC según tipo de RCP}

[Figura del notebook mostrando las tasas de ROSC]

\subsubsection{Supervivencia a 7 Días}

[Los resultados de supervivencia se insertarán aquí]

\textbf{Figura 2. Supervivencia a 7 días según RCP transtelefónica}

[Figura correspondiente del notebook]

\subsection{Análisis Multivariante}

\textbf{Tabla 2. Análisis de regresión logística multivariante para ROSC}

[Tabla con OR, IC 95% y valores p]

\textbf{Tabla 3. Análisis de regresión logística multivariante para supervivencia a 7 días}

[Tabla correspondiente]

\subsection{Análisis Estratificado}

\subsubsection{Efectividad según Tiempo de Respuesta}

[Análisis de la efectividad de la RCP transtelefónica estratificado por tiempo de llegada]

\subsubsection{Efectividad según Características Demográficas}

[Análisis estratificado por edad y sexo]

\section{Discusión}

[Esta sección se desarrollará interpretando los resultados obtenidos en el contexto de la literatura científica existente]

\subsection{Interpretación de los Resultados Principales}

[Discusión de los hallazgos principales]

\subsection{Comparación con la Literatura}

[Comparación con estudios previos]

\subsection{Implicaciones Clínicas}

[Relevancia para la práctica clínica]

\subsection{Limitaciones del Estudio}

Este estudio presenta varias limitaciones que deben considerarse en la interpretación de los resultados:

\begin{itemize}
\item Diseño retrospectivo que limita el control de variables de confusión no registradas
\item Posible sesgo de selección inherente a los registros administrativos
\item Variabilidad en la calidad y completitud de los registros
\item Imposibilidad de evaluar la calidad real de la RCP realizada por testigos
\item Limitación temporal del seguimiento a 7 días
\end{itemize}

\section{Conclusiones}

[Las conclusiones se desarrollarán basándose en los resultados del análisis]

\section{Agradecimientos}

Los autores agradecen al personal de SAMUR-PC por su dedicación en la atención de emergencias y en el registro de datos que han hecho posible este estudio.

\section{Conflicto de Intereses}

Los autores declaran no tener conflictos de intereses en relación con este estudio.

\section{Financiación}

Este estudio no recibió financiación específica.

\begin{thebibliography}{99}

\bibitem{ref1} Gräsner JT, Herlitz J, Tjelmeland IBM, et al. European Resuscitation Council Guidelines 2021: Epidemiology of cardiac arrest in Europe. Resuscitation. 2021;161:61-79.

\bibitem{ref2} Yan S, Gan Y, Jiang N, et al. The global survival rate among adult out-of-hospital cardiac arrest patients who received cardiopulmonary resuscitation: a systematic review and meta-analysis. Crit Care. 2020;24(1):61.

\bibitem{ref3} Pasdfasd

\bibitem{ref4} Hasselqvist-Ax I, Riva G, Herlitz J, et al. Early cardiopulmonary resuscitation in out-of-hospital cardiac arrest. N Engl J Med. 2015;372(24):2307-2315.

\bibitem{ref5} Ro YS, Shin SD, Lee YJ, et al. Effect of Dispatcher-Assisted Cardiopulmonary Resuscitation Program and Location of Out-of-Hospital Cardiac Arrest on Survival and Neurologic Outcome. Ann Emerg Med. 2017;69(1):52-61.

\end{thebibliography}

\end{multicols}

\end{document}